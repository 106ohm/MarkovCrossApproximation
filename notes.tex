\documentclass[]{article}

%opening
\title{Chebyshev Approximation based Solution\\of Markov Dependability Models}
\author{}

\usepackage{amsmath}
\usepackage{amsfonts}
\usepackage{bbm}
\usepackage{cleveref}
\usepackage{url}

\begin{document}

\maketitle

\begin{abstract}

\end{abstract}

\section{Problem statement}
We want to find the solution bundle $\pi(t,\theta)$ of the system of PDEs
\begin{equation}
\label{eq:PDEs}
\begin{cases}
\frac{\partial\pi(t,\theta)}{\partial t} &= Q(\theta)\pi(t,\theta),\\
\pi(0,\theta) &= \pi_0,
\end{cases}
\end{equation}
where $\pi_0\in\mathbb R^N$, $\theta\in\mathbb R^p$, 
$Q(\theta)\in\mathbb R^{N\times N}$, $t\in[0,t_f]$, and 
$\theta_i\in [l_i,u_i]$.
This system originates from the Kolmogorov Forward Equation (a system of ODEs)
$d\pi_{\theta}(t)/dt = Q_{\theta}\pi_{\theta}(t)$,
where the only independent variable is $t$, and $\theta$ is a vector of constants 
jet to be defined, i.e., parameters, that can be used to label the solutions.
If $\theta$ is upgraded to the independent variable status, i.e., when we 
are looking for the solution bundle $\pi(t,p)$, 
we obtain the system of PDEs defined in \Cref{eq:PDEs}.

\section{Observations}
First trial: exploit Chebfun\footnote{\url{https://www.chebfun.org}}.
In particular \texttt{chebop} and \texttt{chebop2}.
With \texttt{chebop} we can address the Kolmogorov Forward Equation, where $\theta$ 
are constants, because the problem is stated as the approximation of the 
solution of a system of linear ODEs.
We can address $\pi_{\theta}$ as a vector, the problem is that we cannot 
address $\theta$ as a vector of independent variables.

With \texttt{chebop2} we can address a single additional independent variable, 
i.e., $\theta\in\mathbb R$, and this is quite restrictive by itself,
but the real issue is that we cannot address $\pi$ as a vector.

\section{Approximation ``by hand''}
First consider the Kolmogorov Forward equations (system pf ODEs), i.e.,
$\theta$ is a constant, and actually we can avoid mentioning it.
Calling $T_j$ the $j$-th Chebyshev polynomials, we can define
\begin{equation}
\label{eq:hatpi1d}
\hat{\pi}_i(t) = \sum_{j=1}^n X_{ij}T_j(\varphi(t)),
\end{equation}
where $\varphi$ sends $[0,t_f]$ to $[-1,1]$, and $X\in\mathbb R^{N,n}$ is the
unknown matrix.
Observing that the $a_h$ in $T_j'=\sum_{h=0}^{j-1}a_hT_h$ are well-known constants,
defining $A_{jh}=\mathbbm{1}_{\{h<j\}}a_h$ so that $A$ is triangular, we can write
\[
\forall i. \sum_{j,h}X_{ij}A_{jh}\varphi'(t)T_h(\varphi(t))
-\sum_{l,j}Q_{il}X_{lj}T_j(\varphi(t))\approx 0\in\mathbb R,
\] 
i.e., $\forall i.X_iADT-Q_iXT\approx 0$, calling $X_i$ the $i$-th row of $X$,
$T\mathbb R^{n}$ the vector whose entries are $T_j(\varphi(t_h))$,
$t_h$ the node points, and $D$ is the diagonal matrix whose entries are
equal to $\varphi'(t_h)$. 
Then $(XDA-QX)T\approx 0\in\mathbb R^N$, that resemble a Sylvester equation.
When the initial conditions $\pi_0$ are included within the picture,
we can hope to solve this equation. 

Notice that\footnote{\url{https://arxiv.org/pdf/1409.2789.pdf}} 
in \texttt{chebop}, instead of writing $T'_j$ in terms of Chebyshev polynomial, 
the ultraspherical polynomials are employed, so that leading to a sparse matrix
instead of a triangolar one, but for now we want to simplify things as much as possible.

In \texttt{chebop2} a discretization of a separable representation of the
differential operator is considered. Here instead we have to deal just with a
1D partial derivative, and in addition the right hand side of the PDE has
a simple shape. Thus, we want to do computation ``by hand''.
Define the approximation to the solution bundle as
\begin{equation}
\label{eq:hatpi}
\hat{\pi}_i(t) = \sum_{j,k_1,\dots,k_p=1}^n 
X_{ijk_1\dots k_p}T_j(\varphi(t))
T_{k_1}(\psi_i(\theta_1))\cdot T_{k_p}(\psi_p(\theta_p)),
\end{equation}
where $\psi_i$ sends $[l_i,u_i]$ to $[-1,1]$. We have
\begin{align*}
\forall i. &\sum_{j,h,,k_1,\dots,k_p}X_{ij}A_{jh}\varphi'(t)T_h(\varphi(t))
T_{k_1}(\psi_i(\theta_1))\cdot T_{k_p}(\psi_p(\theta_p))+\\
-&\sum_{l,j,,k_1,\dots,k_p}Q_{il}X_{lj}T_j(\varphi(t))T_{k_1}(\psi_i(\theta_1))\cdot T_{k_p}(\psi_p(\theta_p))
\approx 0\in\mathbb R,
\end{align*}
but this time things are not-so-nice as before.


\end{document}
